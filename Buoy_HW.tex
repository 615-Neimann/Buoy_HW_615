% Options for packages loaded elsewhere
\PassOptionsToPackage{unicode}{hyperref}
\PassOptionsToPackage{hyphens}{url}
%
\documentclass[
]{article}
\usepackage{amsmath,amssymb}
\usepackage{iftex}
\ifPDFTeX
  \usepackage[T1]{fontenc}
  \usepackage[utf8]{inputenc}
  \usepackage{textcomp} % provide euro and other symbols
\else % if luatex or xetex
  \usepackage{unicode-math} % this also loads fontspec
  \defaultfontfeatures{Scale=MatchLowercase}
  \defaultfontfeatures[\rmfamily]{Ligatures=TeX,Scale=1}
\fi
\usepackage{lmodern}
\ifPDFTeX\else
  % xetex/luatex font selection
\fi
% Use upquote if available, for straight quotes in verbatim environments
\IfFileExists{upquote.sty}{\usepackage{upquote}}{}
\IfFileExists{microtype.sty}{% use microtype if available
  \usepackage[]{microtype}
  \UseMicrotypeSet[protrusion]{basicmath} % disable protrusion for tt fonts
}{}
\makeatletter
\@ifundefined{KOMAClassName}{% if non-KOMA class
  \IfFileExists{parskip.sty}{%
    \usepackage{parskip}
  }{% else
    \setlength{\parindent}{0pt}
    \setlength{\parskip}{6pt plus 2pt minus 1pt}}
}{% if KOMA class
  \KOMAoptions{parskip=half}}
\makeatother
\usepackage{xcolor}
\usepackage[margin=1in]{geometry}
\usepackage{color}
\usepackage{fancyvrb}
\newcommand{\VerbBar}{|}
\newcommand{\VERB}{\Verb[commandchars=\\\{\}]}
\DefineVerbatimEnvironment{Highlighting}{Verbatim}{commandchars=\\\{\}}
% Add ',fontsize=\small' for more characters per line
\usepackage{framed}
\definecolor{shadecolor}{RGB}{248,248,248}
\newenvironment{Shaded}{\begin{snugshade}}{\end{snugshade}}
\newcommand{\AlertTok}[1]{\textcolor[rgb]{0.94,0.16,0.16}{#1}}
\newcommand{\AnnotationTok}[1]{\textcolor[rgb]{0.56,0.35,0.01}{\textbf{\textit{#1}}}}
\newcommand{\AttributeTok}[1]{\textcolor[rgb]{0.13,0.29,0.53}{#1}}
\newcommand{\BaseNTok}[1]{\textcolor[rgb]{0.00,0.00,0.81}{#1}}
\newcommand{\BuiltInTok}[1]{#1}
\newcommand{\CharTok}[1]{\textcolor[rgb]{0.31,0.60,0.02}{#1}}
\newcommand{\CommentTok}[1]{\textcolor[rgb]{0.56,0.35,0.01}{\textit{#1}}}
\newcommand{\CommentVarTok}[1]{\textcolor[rgb]{0.56,0.35,0.01}{\textbf{\textit{#1}}}}
\newcommand{\ConstantTok}[1]{\textcolor[rgb]{0.56,0.35,0.01}{#1}}
\newcommand{\ControlFlowTok}[1]{\textcolor[rgb]{0.13,0.29,0.53}{\textbf{#1}}}
\newcommand{\DataTypeTok}[1]{\textcolor[rgb]{0.13,0.29,0.53}{#1}}
\newcommand{\DecValTok}[1]{\textcolor[rgb]{0.00,0.00,0.81}{#1}}
\newcommand{\DocumentationTok}[1]{\textcolor[rgb]{0.56,0.35,0.01}{\textbf{\textit{#1}}}}
\newcommand{\ErrorTok}[1]{\textcolor[rgb]{0.64,0.00,0.00}{\textbf{#1}}}
\newcommand{\ExtensionTok}[1]{#1}
\newcommand{\FloatTok}[1]{\textcolor[rgb]{0.00,0.00,0.81}{#1}}
\newcommand{\FunctionTok}[1]{\textcolor[rgb]{0.13,0.29,0.53}{\textbf{#1}}}
\newcommand{\ImportTok}[1]{#1}
\newcommand{\InformationTok}[1]{\textcolor[rgb]{0.56,0.35,0.01}{\textbf{\textit{#1}}}}
\newcommand{\KeywordTok}[1]{\textcolor[rgb]{0.13,0.29,0.53}{\textbf{#1}}}
\newcommand{\NormalTok}[1]{#1}
\newcommand{\OperatorTok}[1]{\textcolor[rgb]{0.81,0.36,0.00}{\textbf{#1}}}
\newcommand{\OtherTok}[1]{\textcolor[rgb]{0.56,0.35,0.01}{#1}}
\newcommand{\PreprocessorTok}[1]{\textcolor[rgb]{0.56,0.35,0.01}{\textit{#1}}}
\newcommand{\RegionMarkerTok}[1]{#1}
\newcommand{\SpecialCharTok}[1]{\textcolor[rgb]{0.81,0.36,0.00}{\textbf{#1}}}
\newcommand{\SpecialStringTok}[1]{\textcolor[rgb]{0.31,0.60,0.02}{#1}}
\newcommand{\StringTok}[1]{\textcolor[rgb]{0.31,0.60,0.02}{#1}}
\newcommand{\VariableTok}[1]{\textcolor[rgb]{0.00,0.00,0.00}{#1}}
\newcommand{\VerbatimStringTok}[1]{\textcolor[rgb]{0.31,0.60,0.02}{#1}}
\newcommand{\WarningTok}[1]{\textcolor[rgb]{0.56,0.35,0.01}{\textbf{\textit{#1}}}}
\usepackage{graphicx}
\makeatletter
\def\maxwidth{\ifdim\Gin@nat@width>\linewidth\linewidth\else\Gin@nat@width\fi}
\def\maxheight{\ifdim\Gin@nat@height>\textheight\textheight\else\Gin@nat@height\fi}
\makeatother
% Scale images if necessary, so that they will not overflow the page
% margins by default, and it is still possible to overwrite the defaults
% using explicit options in \includegraphics[width, height, ...]{}
\setkeys{Gin}{width=\maxwidth,height=\maxheight,keepaspectratio}
% Set default figure placement to htbp
\makeatletter
\def\fps@figure{htbp}
\makeatother
\setlength{\emergencystretch}{3em} % prevent overfull lines
\providecommand{\tightlist}{%
  \setlength{\itemsep}{0pt}\setlength{\parskip}{0pt}}
\setcounter{secnumdepth}{-\maxdimen} % remove section numbering
\ifLuaTeX
  \usepackage{selnolig}  % disable illegal ligatures
\fi
\usepackage{bookmark}
\IfFileExists{xurl.sty}{\usepackage{xurl}}{} % add URL line breaks if available
\urlstyle{same}
\hypersetup{
  pdftitle={Buoy\_HW},
  pdfauthor={Jonathan Neimann},
  hidelinks,
  pdfcreator={LaTeX via pandoc}}

\title{Buoy\_HW}
\author{Jonathan Neimann}
\date{2024-09-27}

\begin{document}
\maketitle

\begin{Shaded}
\begin{Highlighting}[]
\FunctionTok{library}\NormalTok{(data.table)}
\FunctionTok{library}\NormalTok{(lubridate)}
\FunctionTok{library}\NormalTok{(dplyr)}
\FunctionTok{library}\NormalTok{(ggplot2)}

\CommentTok{\#Question A}

\NormalTok{file\_root }\OtherTok{=} \StringTok{"https://www.ndbc.noaa.gov/view\_text\_file.php?filename=44013h"}
\NormalTok{years }\OtherTok{=} \DecValTok{1985}\SpecialCharTok{:}\DecValTok{2023}
\NormalTok{tail }\OtherTok{=} \StringTok{".txt.gz\&dir=data/historical/stdmet/"}
\NormalTok{bd\_list }\OtherTok{=} \FunctionTok{list}\NormalTok{()}

\CommentTok{\#loop through the years and skip lines based on data}

\ControlFlowTok{for}\NormalTok{(year }\ControlFlowTok{in}\NormalTok{ years)\{}
\NormalTok{  path}\OtherTok{\textless{}{-}}\FunctionTok{paste0}\NormalTok{(file\_root,year,tail)}
\NormalTok{  header}\OtherTok{=}\FunctionTok{scan}\NormalTok{(path,}\AttributeTok{what=} \StringTok{\textquotesingle{}character\textquotesingle{}}\NormalTok{,}\AttributeTok{nlines=}\DecValTok{1}\NormalTok{)}
\NormalTok{  skip\_lines }\OtherTok{=} \FunctionTok{ifelse}\NormalTok{(year }\SpecialCharTok{\textgreater{}=} \DecValTok{2007}\NormalTok{, }\DecValTok{2}\NormalTok{, }\DecValTok{1}\NormalTok{)}
\NormalTok{  buoy}\OtherTok{\textless{}{-}}\FunctionTok{fread}\NormalTok{(path,}\AttributeTok{header=}\ConstantTok{FALSE}\NormalTok{,}\AttributeTok{skip=}\NormalTok{skip\_lines)}
\NormalTok{  num\_col }\OtherTok{=} \FunctionTok{ncol}\NormalTok{(buoy)}
  \ControlFlowTok{if}\NormalTok{ (}\FunctionTok{length}\NormalTok{(header) }\SpecialCharTok{\textgreater{}}\NormalTok{ num\_col)\{}
\NormalTok{    header }\OtherTok{=}\NormalTok{ header[}\DecValTok{1}\SpecialCharTok{:}\NormalTok{num\_col]}
\NormalTok{  \} }\ControlFlowTok{else} \ControlFlowTok{if}\NormalTok{ (}\FunctionTok{length}\NormalTok{(header) }\SpecialCharTok{\textless{}}\NormalTok{ num\_col) \{}
\NormalTok{    header }\OtherTok{=} \FunctionTok{c}\NormalTok{(header, }\FunctionTok{paste0}\NormalTok{(}\StringTok{"V"}\NormalTok{,(}\FunctionTok{length}\NormalTok{(header) }\SpecialCharTok{+} \DecValTok{1}\NormalTok{)}\SpecialCharTok{:}\NormalTok{num\_col))}
\NormalTok{  \}}
  
  \CommentTok{\#create header}
  \FunctionTok{colnames}\NormalTok{(buoy) }\OtherTok{=}\NormalTok{ header}
  
  \ControlFlowTok{if}\NormalTok{ (}\StringTok{\textquotesingle{}YY\textquotesingle{}} \SpecialCharTok{\%in\%} \FunctionTok{colnames}\NormalTok{(buoy)) \{}
\NormalTok{    buoy}\SpecialCharTok{$}\NormalTok{YY }\OtherTok{=} \FunctionTok{ifelse}\NormalTok{(buoy}\SpecialCharTok{$}\NormalTok{YY }\SpecialCharTok{\textless{}} \DecValTok{100}\NormalTok{, }\FunctionTok{ifelse}\NormalTok{(buoy}\SpecialCharTok{$}\NormalTok{YY }\SpecialCharTok{\textgreater{}} \DecValTok{20}\NormalTok{, }\DecValTok{1900} \SpecialCharTok{+}\NormalTok{ buoy}\SpecialCharTok{$}\NormalTok{YY, }\DecValTok{2000} \SpecialCharTok{+}\NormalTok{ buoy}\SpecialCharTok{$}\NormalTok{YY), buoy}\SpecialCharTok{$}\NormalTok{YY)}
\NormalTok{  \}}
  
  \ControlFlowTok{if}\NormalTok{ (}\StringTok{\textquotesingle{}YY\textquotesingle{}} \SpecialCharTok{\%in\%} \FunctionTok{colnames}\NormalTok{(buoy) }\SpecialCharTok{\&} \StringTok{"MM"} \SpecialCharTok{\%in\%} \FunctionTok{colnames}\NormalTok{(buoy) }\SpecialCharTok{\&} \StringTok{\textquotesingle{}DD\textquotesingle{}} \SpecialCharTok{\%in\%} \FunctionTok{colnames}\NormalTok{(buoy) }\SpecialCharTok{\&} \StringTok{\textquotesingle{}hh\textquotesingle{}} \SpecialCharTok{\%in\%} \FunctionTok{colnames}\NormalTok{(buoy) }\SpecialCharTok{\&} \StringTok{\textquotesingle{}mm\textquotesingle{}} \SpecialCharTok{\%in\%} \FunctionTok{colnames}\NormalTok{(buoy)) \{}
\NormalTok{    buoy}\SpecialCharTok{$}\NormalTok{Date }\OtherTok{=} \FunctionTok{ymd\_hms}\NormalTok{(}\FunctionTok{paste}\NormalTok{(buoy}\SpecialCharTok{$}\NormalTok{YY, buoy}\SpecialCharTok{$}\NormalTok{MM, buoy}\SpecialCharTok{$}\NormalTok{DD, buoy}\SpecialCharTok{$}\NormalTok{hh, buoy}\SpecialCharTok{$}\NormalTok{mm))}
\NormalTok{  \}}
  
\NormalTok{  bd\_list[[}\FunctionTok{as.character}\NormalTok{(year)]] }\OtherTok{=}\NormalTok{ buoy}
\NormalTok{\}}
\end{Highlighting}
\end{Shaded}

\begin{verbatim}
## Warning in fread(path, header = FALSE, skip = skip_lines): Stopped early on
## line 5114. Expected 16 fields but found 17. Consider fill=TRUE and
## comment.char=. First discarded non-empty line: <<2000 08 01 00 78 4.3 5.1 0.58
## 8.33 5.36 999 1022.9 17.3 17.5 15.0 99.0 99.00>>
\end{verbatim}

\begin{Shaded}
\begin{Highlighting}[]
\NormalTok{bd\_list }\OtherTok{=} \FunctionTok{rbindlist}\NormalTok{(bd\_list, }\AttributeTok{fill =} \ConstantTok{TRUE}\NormalTok{)}

\CommentTok{\#Create Date Column and get rid of excess columns}
\NormalTok{bd\_list }\OtherTok{=}\NormalTok{ bd\_list }\SpecialCharTok{\%\textgreater{}\%}
  \FunctionTok{mutate}\NormalTok{(}\AttributeTok{Year =} \FunctionTok{coalesce}\NormalTok{(}\FunctionTok{as.numeric}\NormalTok{(YY), }\FunctionTok{as.numeric}\NormalTok{(YYYY), }\FunctionTok{as.numeric}\NormalTok{(}\StringTok{\textasciigrave{}}\AttributeTok{\#YY}\StringTok{\textasciigrave{}}\NormalTok{))) }\SpecialCharTok{\%\textgreater{}\%}
  \FunctionTok{select}\NormalTok{(}\SpecialCharTok{{-}}\NormalTok{YY, }\SpecialCharTok{{-}}\NormalTok{YYYY, }\SpecialCharTok{{-}}\StringTok{\textasciigrave{}}\AttributeTok{\#YY}\StringTok{\textasciigrave{}}\NormalTok{) }\SpecialCharTok{\%\textgreater{}\%}
  \FunctionTok{select}\NormalTok{(Year, }\FunctionTok{everything}\NormalTok{())}

\NormalTok{bd\_list }\OtherTok{=}\NormalTok{ bd\_list }\SpecialCharTok{\%\textgreater{}\%}
  \FunctionTok{mutate}\NormalTok{(}\AttributeTok{Wind\_Direction =} \FunctionTok{coalesce}\NormalTok{(WD, WDIR)) }\SpecialCharTok{\%\textgreater{}\%}
  \FunctionTok{select}\NormalTok{(}\SpecialCharTok{{-}}\NormalTok{WD, }\SpecialCharTok{{-}}\NormalTok{WDIR)}

\NormalTok{bd\_list }\OtherTok{=}\NormalTok{ bd\_list }\SpecialCharTok{\%\textgreater{}\%}
  \FunctionTok{mutate}\NormalTok{(}\AttributeTok{Pressure =} \FunctionTok{coalesce}\NormalTok{(BAR, PRES)) }\SpecialCharTok{\%\textgreater{}\%}
  \FunctionTok{select}\NormalTok{(}\SpecialCharTok{{-}}\NormalTok{BAR, }\SpecialCharTok{{-}}\NormalTok{PRES)}

\NormalTok{bd\_list}\SpecialCharTok{$}\NormalTok{Date }\OtherTok{\textless{}{-}} \FunctionTok{ymd\_h}\NormalTok{(}\FunctionTok{paste}\NormalTok{(bd\_list}\SpecialCharTok{$}\NormalTok{Year, bd\_list}\SpecialCharTok{$}\NormalTok{MM, bd\_list}\SpecialCharTok{$}\NormalTok{DD, bd\_list}\SpecialCharTok{$}\NormalTok{hh, }\AttributeTok{sep =} \StringTok{"{-}"}\NormalTok{))}
\NormalTok{bd\_list }\OtherTok{\textless{}{-}}\NormalTok{ bd\_list[, }\SpecialCharTok{{-}}\FunctionTok{c}\NormalTok{(}\DecValTok{1}\SpecialCharTok{:}\DecValTok{4}\NormalTok{)]}
\NormalTok{bd\_list }\OtherTok{\textless{}{-}}\NormalTok{ bd\_list[, }\FunctionTok{c}\NormalTok{(}\DecValTok{15}\NormalTok{, }\DecValTok{1}\SpecialCharTok{:}\DecValTok{14}\NormalTok{)]}
\end{Highlighting}
\end{Shaded}

\#Question B

I think most of time it is appropriate to convert missing data to NA.
This is because there are built in R functions that will allow you to
omitt them when running statistic calculations on the data. If the cell
is simply blank, finding things like the mean and the median of certain
variables would not work. However, there are some situations where
converting them to NA's might not be applicable. For example, maybe a
missing cell is serving as a placeholder for data that will be collected
at another dat, or maybe certain inputs (ie 999) mean something to the
data collector and represent a key that we are unaware of.

For this data set, there seems to be a lot of NA''s when we replace the
99's and 999's it's hard to determine any patterns other than some years
had certain data collected while other years did not (if you add up the
sum of qall the NA's in the data frame it is 2,314,343). One thing you
can see though is that the more recent the dates get, the more data
seems to be filled in, indicating that the buoy has been upgraded over
time. If there is a section that was once collecting data and is no
longer, maybe that part of the buoy broke or malfunctioned and was fixed
at a later date. Below is the code i used to replace all 99's and 999's
with NA's in the data frame.

\begin{Shaded}
\begin{Highlighting}[]
\NormalTok{bd\_list[bd\_list }\SpecialCharTok{==} \DecValTok{99} \SpecialCharTok{|}\NormalTok{ bd\_list }\SpecialCharTok{==} \DecValTok{999}\NormalTok{] }\OtherTok{\textless{}{-}} \ConstantTok{NA}
\end{Highlighting}
\end{Shaded}

\#Question C

Yes this buoy data can show trends of climate change in both air
temperature and water temperature data. First we can look at air
temperature. What i did here was create a new data frame called
buoy\_year\_air which shrinks down the dates to just one row per each
ear and then averages out the average air temperature for that year. I
then plotted it out using a simple scatterplot with years on the x axis
and average temperature on the y axis and connected them with a line to
show the trend. Although there is an upward overall trend, i was
somewhat surprised to see it fairly level throughout all the years, both
rising and falling fairly consistently. There is however, 2 huge spikes
in average in both 2012 and 2023. This data may be a bit misleading
however as the buoy is taking the average air temperature of it's
location over water, where air temperature can remain rather temperate.
Where we really start to see a bigger indicator of global warming is
when we look at the average water temperature. (note* the year 2022 came
up as NA for it's average air temperature so I excluded it from the
plot).

\begin{Shaded}
\begin{Highlighting}[]
\NormalTok{buoy\_year\_air }\OtherTok{\textless{}{-}}\NormalTok{ bd\_list }\SpecialCharTok{\%\textgreater{}\%}
  \FunctionTok{mutate}\NormalTok{(}\AttributeTok{Year =} \FunctionTok{year}\NormalTok{(Date)) }\SpecialCharTok{\%\textgreater{}\%}
  \FunctionTok{group\_by}\NormalTok{(Year) }\SpecialCharTok{\%\textgreater{}\%}
  \FunctionTok{summarize}\NormalTok{(}\AttributeTok{Avg\_ATMP =} \FunctionTok{mean}\NormalTok{(ATMP, }\AttributeTok{na.rm =} \ConstantTok{TRUE}\NormalTok{), }\AttributeTok{.groups =} \StringTok{\textquotesingle{}drop\textquotesingle{}}\NormalTok{) }\SpecialCharTok{\%\textgreater{}\%}
  \FunctionTok{filter}\NormalTok{(}\SpecialCharTok{!}\FunctionTok{is.na}\NormalTok{(Avg\_ATMP))  }

\FunctionTok{ggplot}\NormalTok{(buoy\_year\_air, }\FunctionTok{aes}\NormalTok{(}\AttributeTok{x =}\NormalTok{ Year, }\AttributeTok{y =}\NormalTok{ Avg\_ATMP)) }\SpecialCharTok{+}
  \FunctionTok{geom\_line}\NormalTok{(}\AttributeTok{color =} \StringTok{"blue"}\NormalTok{) }\SpecialCharTok{+}
  \FunctionTok{geom\_point}\NormalTok{() }\SpecialCharTok{+}
  \FunctionTok{labs}\NormalTok{(}\AttributeTok{title =} \StringTok{"Average Air Temperature Over Time"}\NormalTok{,}
       \AttributeTok{x =} \StringTok{"Year"}\NormalTok{,}
       \AttributeTok{y =} \StringTok{"Average Air Temperature (°C)"}\NormalTok{) }\SpecialCharTok{+}
  \FunctionTok{theme\_minimal}\NormalTok{()}
\end{Highlighting}
\end{Shaded}

\includegraphics{Buoy_HW_files/figure-latex/unnamed-chunk-3-1.pdf}

Here in the water temperature graph (constructed the same way) we can
see a clear upward trend in average temperature from 1985 to 2023,
indicating the effects of climate change in the oceans. It is worth
noting that in both graphs, the year2 1997 and 2012 seem t have
significant spikes in either direction. This may be worth examining more
with further EDA.

\begin{Shaded}
\begin{Highlighting}[]
\NormalTok{buoy\_year\_water }\OtherTok{\textless{}{-}}\NormalTok{ bd\_list }\SpecialCharTok{\%\textgreater{}\%}
  \FunctionTok{mutate}\NormalTok{(}\AttributeTok{Year =} \FunctionTok{year}\NormalTok{(Date)) }\SpecialCharTok{\%\textgreater{}\%}
  \FunctionTok{group\_by}\NormalTok{(Year) }\SpecialCharTok{\%\textgreater{}\%}
  \FunctionTok{summarize}\NormalTok{(}\AttributeTok{Avg\_WTMP =} \FunctionTok{mean}\NormalTok{(WTMP, }\AttributeTok{na.rm =} \ConstantTok{TRUE}\NormalTok{), }\AttributeTok{.groups =} \StringTok{\textquotesingle{}drop\textquotesingle{}}\NormalTok{) }\SpecialCharTok{\%\textgreater{}\%}
  \FunctionTok{filter}\NormalTok{(}\SpecialCharTok{!}\FunctionTok{is.na}\NormalTok{(Avg\_WTMP)) }

\FunctionTok{ggplot}\NormalTok{(buoy\_year\_water, }\FunctionTok{aes}\NormalTok{(}\AttributeTok{x =}\NormalTok{ Year, }\AttributeTok{y =}\NormalTok{ Avg\_WTMP)) }\SpecialCharTok{+}
  \FunctionTok{geom\_line}\NormalTok{(}\AttributeTok{color =} \StringTok{"blue"}\NormalTok{) }\SpecialCharTok{+}
  \FunctionTok{geom\_point}\NormalTok{() }\SpecialCharTok{+}
  \FunctionTok{labs}\NormalTok{(}\AttributeTok{title =} \StringTok{"Average Water Temperature Over Time"}\NormalTok{,}
       \AttributeTok{x =} \StringTok{"Year"}\NormalTok{,}
       \AttributeTok{y =} \StringTok{"Average Water Temperature (°C)"}\NormalTok{) }\SpecialCharTok{+}
  \FunctionTok{theme\_minimal}\NormalTok{()}
\end{Highlighting}
\end{Shaded}

\includegraphics{Buoy_HW_files/figure-latex/unnamed-chunk-4-1.pdf}

In terms of statistical significance, we can run regression models on
both data frames with average air and water temperature as our target
variable and year as our predictor. When we create these models and
display the results we can see that the slope coefficient for year is
positive, further showing a upward linear trend in both water and air
temperature.

\begin{Shaded}
\begin{Highlighting}[]
\NormalTok{air\_model }\OtherTok{=} \FunctionTok{lm}\NormalTok{(Avg\_ATMP}\SpecialCharTok{\textasciitilde{}}\NormalTok{Year, }\AttributeTok{data =}\NormalTok{ buoy\_year\_air)}
\NormalTok{water\_model }\OtherTok{=} \FunctionTok{lm}\NormalTok{(Avg\_WTMP}\SpecialCharTok{\textasciitilde{}}\NormalTok{Year, }\AttributeTok{data =}\NormalTok{ buoy\_year\_water)}

\FunctionTok{summary}\NormalTok{(air\_model)}
\end{Highlighting}
\end{Shaded}

\begin{verbatim}
## 
## Call:
## lm(formula = Avg_ATMP ~ Year, data = buoy_year_air)
## 
## Residuals:
##     Min      1Q  Median      3Q     Max 
## -3.4397 -0.5040  0.0139  0.5168  4.0793 
## 
## Coefficients:
##              Estimate Std. Error t value Pr(>|t|)  
## (Intercept) -74.66171   41.04015  -1.819   0.0772 .
## Year          0.04209    0.02048   2.055   0.0472 *
## ---
## Signif. codes:  0 '***' 0.001 '**' 0.01 '*' 0.05 '.' 0.1 ' ' 1
## 
## Residual standard error: 1.39 on 36 degrees of freedom
## Multiple R-squared:  0.105,  Adjusted R-squared:  0.08013 
## F-statistic: 4.223 on 1 and 36 DF,  p-value: 0.04719
\end{verbatim}

\begin{Shaded}
\begin{Highlighting}[]
\FunctionTok{summary}\NormalTok{(water\_model)}
\end{Highlighting}
\end{Shaded}

\begin{verbatim}
## 
## Call:
## lm(formula = Avg_WTMP ~ Year, data = buoy_year_water)
## 
## Residuals:
##     Min      1Q  Median      3Q     Max 
## -4.5863 -0.2855  0.0819  0.3935  3.9124 
## 
## Coefficients:
##               Estimate Std. Error t value Pr(>|t|)    
## (Intercept) -152.88599   34.68132  -4.408 8.62e-05 ***
## Year           0.08152    0.01731   4.710 3.44e-05 ***
## ---
## Signif. codes:  0 '***' 0.001 '**' 0.01 '*' 0.05 '.' 0.1 ' ' 1
## 
## Residual standard error: 1.216 on 37 degrees of freedom
## Multiple R-squared:  0.3749, Adjusted R-squared:  0.358 
## F-statistic: 22.19 on 1 and 37 DF,  p-value: 3.44e-05
\end{verbatim}

\#Question D

What I am attempting to show here is if air temperature has an effect on
precipitation on a given day. First, I loaded in the rain data, adjusted
the date column to with lubridate and took a look at the summary for the
precipitation variable (HPCP)

\begin{Shaded}
\begin{Highlighting}[]
\FunctionTok{library}\NormalTok{(dplyr)}
\NormalTok{rain\_data }\OtherTok{=} \FunctionTok{read.csv}\NormalTok{(}\StringTok{\textquotesingle{}Rainfall.csv\textquotesingle{}}\NormalTok{)}
\NormalTok{rain\_data}\SpecialCharTok{$}\NormalTok{DATE }\OtherTok{\textless{}{-}} \FunctionTok{ymd\_hm}\NormalTok{(rain\_data}\SpecialCharTok{$}\NormalTok{DATE)}
\FunctionTok{summary}\NormalTok{(rain\_data}\SpecialCharTok{$}\NormalTok{HPCP)}
\end{Highlighting}
\end{Shaded}

\begin{verbatim}
##    Min. 1st Qu.  Median    Mean 3rd Qu.    Max. 
## 0.00000 0.00000 0.01000 0.03875 0.04000 2.03000
\end{verbatim}

We can see that the data for HPCP shows some pretty small number as the
mean and median, with a maximum that is very far away relatively
speaking. This indicates a right skewed distribution which I show here
in a histogram. However since the x axis is so spread apart, I decided
to take the log of HPCP to bring it closer together.

\begin{Shaded}
\begin{Highlighting}[]
\NormalTok{mean\_precipitation }\OtherTok{\textless{}{-}} \FunctionTok{mean}\NormalTok{(rain\_data}\SpecialCharTok{$}\NormalTok{HPCP)}
\FunctionTok{ggplot}\NormalTok{(rain\_data, }\FunctionTok{aes}\NormalTok{(}\AttributeTok{x =} \FunctionTok{log}\NormalTok{(HPCP))) }\SpecialCharTok{+}
  \FunctionTok{geom\_histogram}\NormalTok{(}\AttributeTok{binwidth =} \DecValTok{1}\NormalTok{, }\AttributeTok{fill =} \StringTok{"blue"}\NormalTok{, }\AttributeTok{color =} \StringTok{"white"}\NormalTok{) }\SpecialCharTok{+}
  \FunctionTok{labs}\NormalTok{(}\AttributeTok{title =} \StringTok{"Distribution of Precipitation (HPCP)"}\NormalTok{,}
       \AttributeTok{x =} \StringTok{"Precipitation (log)"}\NormalTok{,}
       \AttributeTok{y =} \StringTok{"Frequency"}\NormalTok{) }\SpecialCharTok{+}
  \FunctionTok{theme\_minimal}\NormalTok{()}
\end{Highlighting}
\end{Shaded}

\begin{verbatim}
## Warning: Removed 11632 rows containing non-finite outside the scale range
## (`stat_bin()`).
\end{verbatim}

\includegraphics{Buoy_HW_files/figure-latex/unnamed-chunk-7-1.pdf}

\begin{Shaded}
\begin{Highlighting}[]
\NormalTok{rain\_data }\OtherTok{\textless{}{-}}\NormalTok{ rain\_data }\SpecialCharTok{\%\textgreater{}\%} \FunctionTok{rename}\NormalTok{(}\AttributeTok{Date =}\NormalTok{ DATE)}
\NormalTok{combined\_data }\OtherTok{\textless{}{-}} \FunctionTok{inner\_join}\NormalTok{(rain\_data, bd\_list, }\AttributeTok{by =} \StringTok{"Date"}\NormalTok{)}
\end{Highlighting}
\end{Shaded}

you can indeed see that this is a right skewed distribution, with the
overwhelming majority of the data in the first three bins (lower amount
of rainfall).

Next i wanted to see the distribution of air temperature, which I also
used a hitagram for.

\begin{Shaded}
\begin{Highlighting}[]
\FunctionTok{ggplot}\NormalTok{(bd\_list, }\FunctionTok{aes}\NormalTok{(}\AttributeTok{x =}\NormalTok{ ATMP)) }\SpecialCharTok{+}
  \FunctionTok{geom\_histogram}\NormalTok{(}\AttributeTok{binwidth =} \DecValTok{1}\NormalTok{, }\AttributeTok{fill =} \StringTok{"orange"}\NormalTok{, }\AttributeTok{color =} \StringTok{"white"}\NormalTok{) }\SpecialCharTok{+}
  \FunctionTok{labs}\NormalTok{(}\AttributeTok{title =} \StringTok{"Distribution of Air Temperature (ATMP)"}\NormalTok{,}
       \AttributeTok{x =} \StringTok{"Temperature (°C)"}\NormalTok{,}
       \AttributeTok{y =} \StringTok{"Frequency"}\NormalTok{) }\SpecialCharTok{+}
  \FunctionTok{theme\_minimal}\NormalTok{()}
\end{Highlighting}
\end{Shaded}

\begin{verbatim}
## Warning: Removed 102761 rows containing non-finite outside the scale range
## (`stat_bin()`).
\end{verbatim}

\includegraphics{Buoy_HW_files/figure-latex/unnamed-chunk-8-1.pdf}

Here you can see the data is a bit more normally distributed, but skewed
slightly to the left. This makes sense based on the trends we saw with
climate change earlier, as higher temperatures would be more prevelant
in the data.

Next I wanted to see the relationship air temperature has on
precipitation, so I combined the two data frames on rows where the dates
matched. Because the rain\_data does not count specifically by every
hour, there are many rows in the buoy data that gor dropped. However the
rain\_data still provided a significant amount of rows to look at and
the join was easily done because we made the date format the same
between the two tables using lubridate

\begin{Shaded}
\begin{Highlighting}[]
\NormalTok{combined\_data }\OtherTok{\textless{}{-}} \FunctionTok{inner\_join}\NormalTok{(rain\_data, bd\_list, }\AttributeTok{by =} \StringTok{"Date"}\NormalTok{)}
\end{Highlighting}
\end{Shaded}

Lastly, I created a scatter plot to show the relationship between air
temperature and precipitation from the new data frame called
combined\_data

\begin{Shaded}
\begin{Highlighting}[]
\FunctionTok{ggplot}\NormalTok{(combined\_data, }\FunctionTok{aes}\NormalTok{(}\AttributeTok{x =}\NormalTok{ ATMP, }\AttributeTok{y =}\NormalTok{ HPCP)) }\SpecialCharTok{+}
  \FunctionTok{geom\_point}\NormalTok{(}\AttributeTok{alpha =} \FloatTok{0.5}\NormalTok{) }\SpecialCharTok{+}
  \FunctionTok{labs}\NormalTok{(}\AttributeTok{title =} \StringTok{"Air Temperature vs. Precipitation"}\NormalTok{,}
       \AttributeTok{x =} \StringTok{"Air Temperature (°C)"}\NormalTok{,}
       \AttributeTok{y =} \StringTok{"Precipitation (inches)"}\NormalTok{) }\SpecialCharTok{+}
  \FunctionTok{theme\_minimal}\NormalTok{()}
\end{Highlighting}
\end{Shaded}

\begin{verbatim}
## Warning: Removed 148 rows containing missing values or values outside the scale range
## (`geom_point()`).
\end{verbatim}

\includegraphics{Buoy_HW_files/figure-latex/unnamed-chunk-10-1.pdf}

As you can see in this plot, there is certainly an increase in the
amount of rain as temperature gets higher, with some pretty huge
precipitation days around the 15-25 degree celcius range. This indicates
that the highest chance of a really big rainfall day would be on a day
that is in this temperature range. It is also worth noting that when the
temperature is above 25 degrees celcius, there is a steep drop off in
precipitation in general. This indicates that on really warm days, it is
not very likely to rain at all.

I decided to run two simple inverse models. One looking at how air
temperature effects precipitation and one on how precipitation effects
air temperature using the combined\_data data frame that I made.

\begin{Shaded}
\begin{Highlighting}[]
\NormalTok{temp\_model1 }\OtherTok{=} \FunctionTok{lm}\NormalTok{(HPCP }\SpecialCharTok{\textasciitilde{}}\NormalTok{ ATMP, }\AttributeTok{data =}\NormalTok{ combined\_data)}
\NormalTok{temp\_model2 }\OtherTok{=} \FunctionTok{lm}\NormalTok{(ATMP }\SpecialCharTok{\textasciitilde{}}\NormalTok{ HPCP, }\AttributeTok{data =}\NormalTok{ combined\_data)}

\FunctionTok{summary}\NormalTok{(temp\_model1)}
\end{Highlighting}
\end{Shaded}

\begin{verbatim}
## 
## Call:
## lm(formula = HPCP ~ ATMP, data = combined_data)
## 
## Residuals:
##      Min       1Q   Median       3Q      Max 
## -0.07032 -0.03588 -0.02249  0.00641  1.97763 
## 
## Coefficients:
##              Estimate Std. Error t value Pr(>|t|)    
## (Intercept) 2.494e-02  7.075e-04   35.25   <2e-16 ***
## ATMP        1.632e-03  6.449e-05   25.31   <2e-16 ***
## ---
## Signif. codes:  0 '***' 0.001 '**' 0.01 '*' 0.05 '.' 0.1 ' ' 1
## 
## Residual standard error: 0.07568 on 29482 degrees of freedom
##   (148 observations deleted due to missingness)
## Multiple R-squared:  0.02127,    Adjusted R-squared:  0.02124 
## F-statistic: 640.8 on 1 and 29482 DF,  p-value: < 2.2e-16
\end{verbatim}

\begin{Shaded}
\begin{Highlighting}[]
\FunctionTok{summary}\NormalTok{(temp\_model2)}
\end{Highlighting}
\end{Shaded}

\begin{verbatim}
## 
## Call:
## lm(formula = ATMP ~ HPCP, data = combined_data)
## 
## Residuals:
##      Min       1Q   Median       3Q      Max 
## -26.7740  -5.1869  -0.7043   5.3654  20.1654 
## 
## Coefficients:
##             Estimate Std. Error t value Pr(>|t|)    
## (Intercept)  8.07400    0.04419  182.71   <2e-16 ***
## HPCP        13.03068    0.51478   25.31   <2e-16 ***
## ---
## Signif. codes:  0 '***' 0.001 '**' 0.01 '*' 0.05 '.' 0.1 ' ' 1
## 
## Residual standard error: 6.762 on 29482 degrees of freedom
##   (148 observations deleted due to missingness)
## Multiple R-squared:  0.02127,    Adjusted R-squared:  0.02124 
## F-statistic: 640.8 on 1 and 29482 DF,  p-value: < 2.2e-16
\end{verbatim}

You can see that the coefficients for the first model are very small,
which makes sense as we are dealing with inches of rain. Yet there is
still a positive slope meaning that everytime the precipitation
increases, we can also expect the temperature to increase ever so
slightly.

The second model tells a similar story but in a different way. Now the
slope coefficient for the HPCP variable is massive. This also makes
sense because it is determining what happens if the precipitation
increases by one full inch, which is a large amount in this data set.
Nnetheless, if that were to happen this model is telling us we can
expect a temperature increase bu 13 degrees celcius.

\end{document}
